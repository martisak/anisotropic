\cleardoublepage
\refstepcounter{dummy}
\addcontentsline{toc}{chapter}{Abstract}
\pagenumbering{roman}\setcounter{page}{1}
\thispagestyle{plain}
\thesistitle\\
% Subtitles go here\\
\textsc{Martin Isaksson}\\
Communication Systems\\
Department of Signals and Systems\\
Chalmers University of Technology\\

{\Huge \bfseries Abstract\par\nobreak
    \vskip 40pt}

%\thispagestyle{empty}
A Wireless Sensor Network (WSN) of anisotropic magnetoresistor sensors offers a low-cost alternative to other traffic measurement technologies. The WSNs offer better reliability, more information, can be deployed quickly and be reused. In this thesis the sensor algorithms used for detection, velocity estimation, queue detection and classification in such a network are evaluated based on simulated and measured data. A number of algorithms are evaluated and the results are compared. A new algorithm for speed estimation using two sensor nodes is proposed and evaluated. It is found to be much better than earlier algorithms, requiring a signal to noise ratio (SNR) of 20 dB less than the traditional algorithm.

\vspace{1.5cm}
\textsc{Keywords:} Wireless Sensor Network, Anisotropic Magnetoresistor, Vehicle Detection, Vehicle Classification

\cleardoublepage
\refstepcounter{dummy}
\addcontentsline{toc}{chapter}{Sammanfattning}
\thesistitle\\
% Subtitles go here\\
\textsc{Martin Isaksson}\\
Communication Systems\\
Department of Signals and Systems\\
Chalmers University of Technology\\

{\Huge \bfseries Sammanfattning\par\nobreak
\vskip 40pt}
\thispagestyle{plain}


Ett tr\aa{}dl\"{o}st sensorn\"{a}tverk (Wireless Sensor Network, WSN) av anisotropiska magnetoresistorsensorer erbjuder ett l\aa{}gkostnadsalternativ till andra trafikm\"{a}tningsmetoder. De ger b\"{a}ttre tillf\"{o}rlitlighet, mer information, kan installeras snabbt och kan \aa{}teranv\"{a}ndas. I det h\"{a}r examensarbetet utv\"{a}rderas sensoralgoritmer f\"{o}r detektion, hastighetsestimering, k\"{o}detektion och klassificering i ett s\aa{}dant n\"{a}tverk baserat p\aa{} simuleringar och m\"{a}tdata. Ett antal algoritmer utv\"{a}rderas och j\"{a}mf\"{o}rs. En ny algoritm f\"{o}r hastighetsestimering f\"{o}rel\aa{}s och evalueras. Det visas att den \"{a}r b\"{a}ttre \"{a}n f\"{o}reg\aa{}ende algoritmer och fordrar ett signal till brus-f\"{o}rh\aa{}llande (SNR) p\aa{} 20 dB mindre \"{a}n tidigare algoritmer.

\vspace{1.5cm}
\textsc{Nyckelord:} Tr\aa{}dl\"{o}sa sensorn\"{a}tverk, anisotropisk magnetoresistor, fordonsdetektion, fordonsklassificering
\newpage